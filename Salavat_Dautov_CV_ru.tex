\documentclass{resume}

\begin{document}

% Заголовок
\begin{center}
  \textbf{\Large Салават Даутов} \\
  \vspace{4pt}
  \small
  \faIcon{envelope}
  \href{mailto:dautovsalavatd@gmail.com}{\underline{dautovsalavat7@gmail.com}} $ $
  \faIcon{telegram}
  \href{https://t.me/sdautov}{\underline{sdautov}} $ $
  \faIcon{github}
  \href{https://github.com/sdautov}{\underline{sdautov}} $ $
  \faIcon{globe}
  \href{https://sdautov.github.io}{\underline{sdautov.github.io}}
\end{center}

% Образование
\section{Образование}
\resumeSubHeadingListStart
\resumeSubheading
{\href{https://www.miet.ru/}{НИУ "Московский институт электронной техники"}}{Москва, Россия}
{09.03.01 Информатика и вычислительная техника, бакалавр}{Август 2018 -- Июнь 2022}
\resumeSubHeadingListEnd

% Опыт работы
\section{Опыт работы}
\resumeSubHeadingListStart
\resumeSubheading
{\href{https://www.latera.ru/}{ООО Латера}}{Москва, Россия}
{Программист-разработчик}{Март 2021 -- Июль 2022}
\resumeItemListStart
\resumeItemPlain{Участвовал в разработке биллинговой системы \href{https://hydra-billing.ru/}{Гидра} и opensource приложения на основе Camunda для автоматизации бизнес процессов \href{https://hydra-oms.com/}{Гидра OMS.}}
\resumeItem{Получил опыт}{
  \resumeItemListStart
  \resumeItemPlain{Написания интеграций с платежными системами.}
  \resumeItemPlain{Разработки BPM на Camunda.}
  \resumeItemPlain{Разработки пакетов, хранимых процедур и функций на PL/SQL.}
  \resumeItemPlain{Оптимизации и написания SQL-запросов.}
  \resumeItemPlain{Работы с CI/CD.}
  \resumeItemPlain{Тестирования с иcпользованием Selenium.}
  \resumeItemPlain{Написания unit-тестов с использованием PyTest Mock.}
  \resumeItemListEnd}
\resumeItem{Использовавшиеся технологии}{Oracle PL/SQL, Ruby on Rails, Bootstrap, Vue.js, Kotlin, React, memcached, Redis.}
\resumeItemListEnd
\resumeSubheading
{ООО НПК Прогресс}{Уфа, Россия}
{Инженер-программист}{Июль 2022 -- Сентябрь 2022}
\resumeItemListStart
\resumeItemPlain{Участвовал в доработках и поддержке информационных систем фонда медицинского страхования Республики Башкортостан. Занимался рефакторингом кода, написанием нового функционала и интеграцией с системой КриптоПро.}
\resumeItem{Использовавшиеся технологии}{Oracle PL/SQL, C\#, ASP.NET Framework, ASP.NET Core, EntityFramework, MongoDB, RabbitMQ.}
\resumeItemListEnd
\resumeSubheading
{\href{https://www.nobilis.team/}{Nobilis.Team}}{Москва, Россия}
{Программист-разработчик}{Сентябрь 2022 -- Февраль 2024}
\resumeItemListStart
\resumeItemPlain{Участвовал в разработке и поддержке решения учета объектов природопользования для Росприроднадзора, а так же в миграции с CRM Salesforce на решение BPMSoft для Philip Morris International. Занимался разработкой бизнес-процессов, написанием клиентской части приложения и серверной логики.}
\resumeItem{Использовавшиеся технологии}{Oracle, PostgreSQL, Microsoft SQL Server, C\#, ASP.NET Framework, ASP.NET Core, Redis.}
\resumeItemListEnd
\resumeSubHeadingListEnd

% Технические навыки
\section{Технические навыки}
\resumeSubHeadingListStart
\resumeItem{Языки программирования}{
  \resumeItemListStart
  \resumeItem{С которыми я имел опыт коммерческой разработки}{С\#, Ruby, Python, SQL, PL/SQL, JavaScript, TypeScript, Kotlin.}
  \resumeItem{Также имею опыт работы с}{С++ и Java.}
  \resumeItemPlain{Изучаю функциональные языки, такие как Haskell и Scala.}
  \resumeItemPlain{Владею навыками верстки HTML и CSS.} 
  \resumeItemListEnd}
\resumeItem{Фреймворки, библиотеки и технологии}{
  \resumeItemListStart
  \resumeItem{Backend}{ASP.NET, EntityFramework, Java EE, JDBC, Hibernate, Ruby on Rails, memcached, Redis.}
  \resumeItem{Frontend}{Vue.js (в том числе опыт миграции с Vue 2 на Vue 3), React, Bootstrap.}
  \resumeItemListEnd}
\resumeSubHeadingListEnd

% Языки
\section{Языки}
\resumeSubHeadingListStart
\resumeItem{Русский}{уровень владения -- родной.}
\resumeItem{Английский}{уровень владения -- Intermediate (B1).}
\resumeSubHeadingListEnd

% Рабочий процесс
\section{Рабочий процесс}
\resumeSubHeadingListStart
\resumeItem{Имею опыт}{
  \resumeItemListStart
  \resumeItemPlain{Коммандной разработки согласно Scrum, Kanban и Adgile.}
  \resumeItemPlain{Работы с регулярными код-ревью.}
  \resumeItemPlain{Ревью чужого кода.}
  \resumeItemPlain{Администрирования серверов на Windows Server, Linux.}
  \resumeItemPlain{Работы с Docker и Kubernetes.}
  \resumeItemPlain{Миграции приложений из Docker в Kubernetes.}
  \resumeItemPlain{Написания документации к коду в Confluence.}
  \resumeItemListEnd}
\resumeItem{Умею работать с}{
  \resumeItemListStart
  \resumeItemPlain{Системами контроля версий, такими как git, SVN, Mercurial.}
  \resumeItemPlain{Системами баг-трекинга, такими как Jira, YouTrack.}
  \resumeItemPlain{Системами непрерывной интеграции Bamboo, GitLab CI/CD, Jenkins.}
  \resumeItemPlain{Брокерами сообщений RabbitMQ, Kafka.}
  \resumeItemPlain{Реляционными базами данных Oracle, Microsoft SQL Server, PostgreSQL, MariaDB, MySQL, Firebird, SQLite.}
  \resumeItemPlain{NoSQL базами данных MongoDB, Redis, memcached.}
  \resumeItemPlain{Такими инструментами, как Swagger, Postman, Grafana Kibana.}
  \resumeItemListEnd}
\resumeItem{Знаком с}{
  \resumeItemListStart
  \resumeItemPlain{Архитектурными паттернами, такими как MVC, MVP, MVVW.}
  \resumeItemPlain{Различными алгоритмами и структурами данных.}
  \resumeItemPlain{Многими паттернами проектирования ООП.}
  \resumeItemPlain{Архитектурными стилями REST, SOAP.}
  \resumeItemPlain{Принципами разработки SOLID, DRY, KISS.}
  \resumeItemListEnd}
\resumeItem{Так же имею опыт разработки}{
  \resumeItemListStart
  \resumeItemPlain{Мобильных приложений Android (в том числе с использованием библиотеки TensorFlow).}
  \resumeItemPlain{Мобильных приложений на Xamarin.}
  \resumeItemPlain{Веб приложения для распознавания изображений с использованием сервиса компьютерного зрения CustomVision.}
  \resumeItemPlain{Desktop приложений с использованием WPF, Swing, Qt.}
  \resumeItemPlain{Мобильной игры с использованием Unity.}
  \resumeItemListEnd}
\resumeSubHeadingListEnd

% Стажировки
\section{Стажировки}
\resumeSubHeadingListStart
\resumeSubheadingShort
{\href{https://safeboard.kaspersky.ru/}{Kaspersky SafeBoard 2019}}{Октябрь 2019 -- Декабрь 2019}
\resumeItemListStart
\resumeItemPlain{Обучался на стриме разработки в Лаборатории Касперского. В процессе обучения написал на ASP.NET несколько веб-сервисов.}
\resumeItem{Использовавшиеся технологии}{Microsoft SQL Server, C\#, ASP.NET Framework, ASP.NET Core, EntityFramework, WPF.}
\resumeItemListEnd
\resumeSubHeadingListEnd

% Онлайн курсы
\section{Онлайн курсы}
\resumeSubHeadingListStart
\resumeSubheadingShort
{\href{https://ru.coursera.org/specializations/c-plus-plus-modern-development}{Искусство разработки на современном C++}}{Сентябрь 2020 -- Февраль 2021}
\resumeItemListStart
\resumeItemPlain{Прошел курсы по разработке на C++ 17 от МФТИ и Яндекса на Coursera.}
\resumeItemListEnd
\resumeSubHeadingListEnd

% Мероприятия
\section{Мероприятия}
\resumeSubHeadingListStart
\resumeItem{Участвовал в различных олимпиадах и хакатонах, среди них}{
  \resumeItemListStart
  \resumeItemPlain{ACM ICPC 2016.}
  \resumeItemPlain{ICPC 2018.}
  \resumeItemPlain{\href{https://microsoft-student-partner.timepad.ru/event/923680/}{Microsoft Student Partners Game Hack 2019} (наша команда получила приз зрительских симпатий).}
  \resumeItemPlain{\href{https://it-mai.timepad.ru/event/934116/}{MAI C\# hackathon 2019} (командный хакатон, в котором наша команда заняла первое место).}
  \resumeItemPlain{ICPC 2019.}
  \resumeItemPlain{\href{https://hack.moscow/}{Hack Moscow 2019}.}
  \resumeItemPlain{Хакатон Kaspersky SafeBoard 2019 (одиночный хакатон в качестве финального этапа отбора).}
  \resumeItemListEnd}
\resumeSubHeadingListEnd

% О себе
\section{О себе}
\resumeSubHeadingListStart
\resumeItemPlain{В 2022 году я успешно окончил НИУ МИЭТ по специальности Информатика и вычислительная техника. В процессе обучения я получил практический опыт программирования одноплатных компьютеров и микроконтроллеров, основанных на архитектуре ARM. В рамках курсовой работы я реализовал процессор на архитектуре RISC-V, используя язык Verilog. Также я разработал "систему на кристалле", что дало мне возможность приобрести опыт работы с различными интерфейсами, такими как VGA, PS/2, USB I2C и SPI.}
\resumeSubHeadingListEnd
\end{document}
